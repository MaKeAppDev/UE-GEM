%
% 
% Geschrieben im WS 2013/2014 an der TU München
% von Markus Hofbauer und Kevin Meyer für LaTeX4EI (latex4ei.de)
% Kontakt: latex@kevin-meyer.de oder via Kontaktformular auf http://latex4ei.de

% Dokumenteinstellungen
% ======================================================================	

\documentclass[10pt,a4paper]{article}
\usepackage[utf8]{inputenc}
\usepackage[german]{babel}
\usepackage{scientific}
\usepackage[european]{circuitikz}
\usepackage{hyperref}

\sisetup{per-mode=fraction,output-decimal-marker={,}}

% Nicht neuen, sondern alten Vector benutzen
\let\newvec = \vec
\let\vec = \oldvec

% Dokumentbeginn
% ======================================================================

\begin{document}

\title{GEM Übung: \textbf{Blatt 6} Mitschrift}
\date{17. Dezember 2013}
\author{Kevin Meyer}
\maketitle

\section*{Zusammenfassung}

% Aufgabe 1
\section{}
Siehe Lösung.

% Aufgabe 2
\section{}
\begin{align*}
u_Y(t) = i_Y(t) \cdot \newvec{Z}\\
i(t) = \hat{I}_{Y} \cdot \sin (\omega t)\\
\newvec Z = R + j \cdot 2\pi \cdot f \cdot L = \SI{1.5}{\ohm} + j \SI{7.854}{\ohm}\\
u_Y(t) = (\SI{1.5}{\ohm} + j \SI{7.854}{\ohm}) \cdot \sqrt{2} \cdot \SI{50}{\ampere} \cdot \sin(\omega t - \phi_i) = (\SI{106.07}{\volt} + j \SI{555.36}{\volt}) \cdot \sin(\omega t - \phi_i)\\
\hat{U}_Y = \sqrt{(\SI{106.07}{\volt})^2 + (\SI{555.36}{\volt})^2} = \SI{565.4}{\volt}\\
\hat{U}_V = \sqrt{3} \cdot \hat{U}_Y = \SI{979.3}{\volt}
\end{align*}

% Aufgabe 3
\section{}
\begin{align*}
S = 3 \cdot U_\text{eff} \cdot I_\text{eff}\\
U_{Y\text{eff}} = \frac{\hat{U}_Y}{\sqrt{2}} = \SI{399.8}{\volt}\\
U_{\triangle\text{eff}} = \frac{\hat{U}_V}{\sqrt{2}} = \SI{692.47}{\volt}\\
I_{Y\text{eff}} = \SI{50}{\ampere}\\
I_{\triangle\text{eff}} = \sqrt{3} \cdot I_{Y\text{eff}} = \SI{86.6}{\ampere} \quad \left( \newvec{Z} = \frac{U_\triangle}{I_\triangle} = \frac{U_Y}{I_Y}\right)\\
S_Y = 3 \cdot U_Y \cdot I_Y = \SI{59.97}{\kilo\volt\ampere}\\
S_\triangle = \SI{179.86}{\kilo\volt\ampere}
\end{align*}

% Aufgabe 4
\section{}
\begin{align*}
I_\triangle = \SI{86.6}{\ampere}\\
U_\triangle = I_\triangle \cdot \newvec Z = \SI{129}{\volt} + j \SI{680.16}{\volt} = \SI{692.45}{\volt} \exp^{j \SI{79.188}{\degree}}\\
P_\triangle = 3 \cdot \Re{U} \cdot I = \SI{33.748}{\kilo\watt}\\
Q_\triangle = 3 \cdot \Im{U} \cdot I = \SI{176.7}{\kilo\text{Var}}\\
\cos(\phi) = \cos(\SI{79.188}{\degree}) = \frac{P_\triangle}{S_\triangle} = 0.19\\
\end{align*}

% Aufgabe 5
\section{}
\begin{align*}
\newvec Z_\text{ges1} = \newvec Z_L + (\newvec Z || 2 \newvec Z) + \newvec Z_L = 2 \newvec Z_L + \frac{\newvec Z \cdot 2 \newvec Z}{\newvec Z + 2 \newvec Z} = 2 \cdot \newvec Z_L + \frac{2}{3} \cdot \newvec Z\\
\newvec Z_\text{ges2} = \left( (\newvec Z + \newvec Z_L^*) || 2(\newvec Z + \newvec Z_L^*) \right) = \frac{2}{3} \cdot (\newvec Z + \newvec Z_L^*) = \frac{2}{3} \newvec Z + \frac{2}{3} \newvec Z_L^*\\
\newvec Z_\text{ges1} = \newvec Z_\text{ges2}\\
2 \newvec Z_L + \frac{2}{3} \newvec Z = \frac{2}{3} \newvec Z_L^* + \frac{2}{3} \newvec Z\\
\newvec Z_L^* = 3 \cdot \newvec Z_L
\end{align*}

% Aufgabe 6
\section{}
\begin{circuitikz}
\draw
	(0,4)
	to [short, i=$\newvec I$, o-] (2,4)
	to [R=$R$] (2,2)
	to [L=$L$] (2,0)
	to [short, -o] (0,0);
\draw
	(0,4) to [open, v>=$\newvec U$] (0,0);
\end{circuitikz}

% Aufgabe 7
\section{}
\begin{align*}
\textbf{Stern } \newvec U = U_Y = \hat{U}_Y \cdot \frac{1}{\sqrt{2}} = \SI{399.8}{\volt}\\
\newvec I = \frac{U}{\newvec Z} = \ldots = \SI{9.38}{\ampere} - j \SI{49.11}{\ampere} = \SI{50}{\ampere} \cdot \exp^{-j \SI{79.188}{\degree}}
\end{align*}

Zeigerdiagramm siehe Lösung.

\end{document}