%
% 
% Geschrieben im WS 2013/2014 an der TU München
% von Markus Hofbauer und Kevin Meyer für LaTeX4EI (latex4ei.de)
% Kontakt: latex@kevin-meyer.de oder via Kontaktformular auf http://latex4ei.de

% Dokumenteinstellungen
% ======================================================================	

\documentclass[10pt,a4paper]{article}
\usepackage[utf8]{inputenc}
\usepackage[german]{babel}
\usepackage{scientific}
\usepackage[european]{circuitikz}
\usepackage{hyperref}

\sisetup{per-mode=fraction,output-decimal-marker={,}}

% Nicht neuen, sondern alten Vector benutzen
\let\newvec = \newvec
\let\newvec = \oldvec

% Dokumentbeginn
% ======================================================================

\begin{document}

\title{GEM Übung: \textbf{Blatt 8} Mitschrift}
\date{21. Januar 2014}
\author{Markus Hofbauer, Kevin Meyer}
\maketitle


% Aufgabe 1
\section{Aufgabe}
\begin{align*}
U_1 = \frac{U_N}{\sqrt{3}} = \SI{6.06}{\kilo\volt}\\
\abs{\vec{I_{N1}}}= \frac{S_N}{3 \cdot U_1} = \SI{605}{\ampere}\\
\abs{\varphi} = \arccos(\SI{-0.8}{}) = \pm \SI{143}{\degree} \Rightarrow \varphi = \SI{-143}{\degree}\\
\vec{I_{N1}} = \SI{605}{\ampere} \cdot e^{j\SI{143}{\degree}}\\
n = \frac{f_1}{p} = \SI{25}{\per\second} = \SI{1500}{\per\minute}\\
P_N = S_N \cdot \cos(\varphi) = \SI{8.8}{\mega\watt}
\end{align*}

% Aufgabe 2
\section{Aufgabe}
\begin{align*}
X_d = 2\pi f \cdot (L_{1h} + L_{1\sigma}) = \SI{15.7}{\ohm}
\end{align*}

% Aufgabe 3
\section{Aufgabe}
\begin{align*}
I_{02} = \frac{U_{ip}(=U_{1N})}{\omega \cdot M_{21} \cdot \sqrt{2}} = \SI{80}{\ampere}\\
I_{K0} = \frac{U_1}{X_d} = \SI{386}{\ampere}
\end{align*}

% Aufgabe 4
\section{Aufgabe}
\begin{align*}
\vec{U_{ipN}} = \vec{U_1} = j X_d \cdot I_{1N} = \SI{14}{\kilo\volt} \cdot e^{j\SI{32.8}{\degree}}
\end{align*}

% Aufgabe 5
\section{Aufgabe}
\begin{align*}
U_{ip} = \omega \cdot M_{21} \cdot \sqrt{2} \cdot I_2\\
I_{N2} = \SI{184.8}{\ampere}\\
\abs{\vec{I_{KIII}}} = \frac{U_{ip}}{X_d} = \SI{891.7}{\ampere}
\end{align*}

% Aufgabe 6
\section{Aufgabe}
\begin{align*}
\underline{U}_1 = \underline{U}_{iP} + jX_d\underline{I}_1\\
\underbrace{\frac{U_1}{jX_d}}_{\underline{I}_{K0}} = \underbrace{\frac{\underline{U}_{iP}}{jX_d}}_{\underline{I}_{KIII}} + \underline{I}_1
\end{align*}

% Aufgabe 7
\section{Aufgabe}
Zeichnung

% Aufgabe 8
\section{Aufgabe}
\begin{itemize}
\item $\underline{I}_{K0}$ bleibt unverändert
\item $\abs{\underline{I}_{KIII}}$ wird kleiner: $I_{KIII8} = \SI{0.8}{}\cdot I_{KIIIN}$
\end{itemize}
Zeichnung

% Aufgabe 9
\section{Aufgabe}
Zeichnung
\begin{itemize}
\item ablesen $I_{KIII9} = \SI{610}{\ampere}$
\item $I_{KIII} \sim U_{iP} = \sim I_2$
\item $I_{29} = \frac{I_{KIII9}}{I_{KIIIN}}\cdot I_{2N} = \ldots = \SI{126.4}{\ampere}$
\end{itemize}

% Aufgabe 10
\section{Aufgabe}
Zeichnung
\begin{itemize}
\item $I_2 =$ konstant $\Rightarrow$ Radius konstant
\item $P_1 = 0$
\end{itemize}

% Aufgabe 11
\section{Aufgabe}
Zeichnung
\begin{itemize}
\item $I_{KIII11} = \SI{470}{\ampere}$ ablesen
\item $I_{211} = \frac{I_{KIII11}}{I_{KIIIN}}\cdot I_{2N} = \SI{97.4}{\ampere}$
\end{itemize}

\end{document}