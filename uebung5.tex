%
% 
% Geschrieben im WS 2013/2014 an der TU München
% von Markus Hofbauer und Kevin Meyer für LaTeX4EI (latex4ei.de)
% Kontakt: latex@kevin-meyer.de oder via Kontaktformular auf http://latex4ei.de

% Dokumenteinstellungen
% ======================================================================	

\documentclass[10pt,a4paper]{article}
\usepackage[utf8]{inputenc}
\usepackage[german]{babel}
\usepackage{scientific}
\usepackage[european]{circuitikz}
\usepackage{hyperref}

\sisetup{per-mode=fraction, locale=DE}
\DeclareSIUnit{\sqrtNm}{\ensuremath{\sqrt{\text{\newton\meter}}}}

% Nicht neuen, sondern alten Vector benutzen
\let\newvec = \vec
\let\vec = \oldvec

% Dokumentbeginn
% ======================================================================

\begin{document}

\title{GEM Übung: \textbf{Blatt 5} Mitschrift}
\date{10. Dezember 2013}
\author{Kevin Meyer}
\maketitle

\section*{Zusammenfassung}
\begin{align*}
I_E & = k_E \cdot I_A\\
\Phi_E & = k_\Phi \cdot I_E\\
& = k_\Phi \cdot k_E \cdot I_A\\
M_i & = k_M \cdot \Phi_E \cdot I_A\\
& = k_M \cdot k_\Phi \cdot k_E \cdot {I_A}^2\\
U_A & = R_{A,\text{res}} \cdot I_A + U_i + U_B\\
U_i & = k_U \cdot \Phi_E \cdot n_m\\
& = k_U \cdot k_\phi \cdot k_E \cdot I_A \cdot n_m\\
M_i & = M_R + M_L\\
R_{A,\text{res}} & = R_V + R_A + R_E || R_P 
\end{align*}

% \pagebreak

% Aufgabe 1
\section{Aufgabe}
\begin{center}
\begin{circuitikz}
\draw [american currents]
	(0,2)
	to [short, i=$I_A$, o-](0.5, 2) 
	to [vR=$R_V$]		(2,2)
	to [R=$R_A$](4,2)
	to [I=$U_i$]			(4,0)
	to [short]			(3,0)
	to [R=$R_E$, i=$I_E$](1,0)
	to [short, -o]		(0,0)
	to [open, v^=$U_A$]	(0,2);
\draw
	(3,0)
	to [short, *-] (3,1)
	to [R=$R_P$] (1,1)
	to [short, -*] (1,0);
\end{circuitikz}
\end{center}

\begin{align*}
R_P = \infty,\ R_V = 0\\
k_\Phi = \unitof{\si{\weber\per\ampere}} = \unitof{\si{\volt\second\per\ampere}}\\
M_i(I_A) = k_M \cdot k_\Phi \cdot k_E \cdot {I_A}^2 = \ldots = \SI{10.94e-3}{\newton\meter\per\ampere\squared} \cdot (k_E) \cdot {I_A}^2\\
n_m = \frac{U_i}{k_U \cdot k_\Phi \cdot k_E \cdot I_A} = \frac{U_A - R_{A, \text{res}} \cdot I_A}{k_U \cdot k_\Phi \cdot k_E \cdot I_A} = \frac{U_A}{k_U \cdot k_\Phi \cdot k_E \cdot I_A} - \frac{R_{A,\text{res}}}{k_U \cdot k_\Phi \cdot k_E}\\
I_A = \sqrt{\frac{M_i}{k_M \cdot k_\Phi \cdot k_E}}\\
n_m = U_A \cdot \sqrt{\frac{1}{2\pi \cdot k_U \cdot k_\Phi \cdot k_E \cdot M_i}} - \frac{R_{A, \text{res}}}{k_U \cdot k_\Phi \cdot k_E}\\
n_m(M_i) = \SI{167.24}{\sqrtNm\per\second} \cdot \frac{1}{\sqrt{M_i}} - \SI{0.87146}{\per\second}
\end{align*}

% Aufgabe 2
\section{Aufgabe}
\begin{align*}
n_m = \SI{40}{\per\second},\ M_i = ?\\
M_i = \ldots = \SI{16.7}{\newton\meter}
\end{align*}

% Aufgabe 3
\section{Aufgabe}
\begin{align*}
\text{Anfahren: } n_m = 0\ \Rightarrow U_i = 0\\
U_A = R_{A,\text{res}} \cdot I_A = (R_V + R_A + R_E) \cdot I_A\\
R_V = \frac{U_A}{I_A} - R_A - R_E\\
I_A = \sqrt{\frac{\SI{112}{\newton\meter}}{\SI{10.94e-3}{\newton\meter\per\ampere\squared}}} = \SI{101.2}{\ampere}\\
R_V = \SI{1.027}{\ohm}
\end{align*}

% Aufgabe 4
\section{Aufgabe}
\begin{align*}
M = \SI{49}{\newton\meter},\ R_V = \SI{1.027}{\ohm},\ n_m = ?\\
n_m = \SI{8.086}{\per\second}
\end{align*}

% Aufgabe 5
\section{Aufgabe}
\begin{align*}
I_E = k_E \cdot I_A = \frac{R_P}{R_P + R_E} \cdot I_A = \SI{0.318}{} \cdot I_A\\
R_{A,\text{res}} = R_A + R_E || R_P = \SI{49}{\milli\ohm}\\
M(I_A) = k_M \cdot k_\Phi \cdot k_E \cdot I_A\\
I_A = \SI{132.33}{\ampere}\\
U_i = U_A - R_{A,\text{res}} \cdot I_A = \SI{103.46}{\volt}\\
n_m = \frac{U_i}{k_U \cdot k_\Phi \cdot k_E \cdot I_A} = \SI{36}{\per\second}
\end{align*}

% Aufgabe 6
\section{Aufgabe}
\begin{align*}
\Phi_E \text{ aus Diagramm}\\
M_i = k_M \cdot \Phi_E \cdot I_A \quad (k_M = \SI{85.94}{})\\
n_m = \frac{U_A - R_{A,\text{res}}}{k_U \cdot \Phi}\\
R_{A,\text{res}} = R_A + R_E = \SI{0.06}{\ohm}\\
k_U = \SI{540}{}
\end{align*}

\begin{tabular}{>{$}c<{$} |>{$}c<{$} >{$}c<{$} >{$}c<{$} >{$}c<{$} >{$}c<{$} >{$}c<{$} >{$}c<{$} >{$}c<{$} >{$}c<{$}}
I_A \unitof{\si{\ampere}} & 0 & 20 & 40 & 60 & 80 & 100 & 120 & 140 & 160 \\ 
\hline 
\Phi &  &  &  &  &  &  &  &  &  \\ 
M_i &  &  &  &  &  &  &  &  &  \\ 
n_m &  &  &  &  &  &  &  &  &  \\ 
\end{tabular} 

\end{document}
