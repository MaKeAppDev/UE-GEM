%
% 
% Geschrieben im WS 2013/2014 an der TU München
% von Markus Hofbauer und Kevin Meyer für LaTeX4EI (latex4ei.de)
% Kontakt: latex@kevin-meyer.de oder via Kontaktformular auf http://latex4ei.de

% Dokumenteinstellungen
% ======================================================================	

\documentclass[10pt,a4paper]{article}
\usepackage[utf8]{inputenc}
\usepackage[german]{babel}
\usepackage{scientific}
\usepackage[european]{circuitikz}

\sisetup{per-mode=fraction,output-decimal-marker={,}}

% Nicht neuen, sondern alten Vector benutzen
\let\newvec = \vec
\let\vec = \oldvec

% Dokumentbeginn
% ======================================================================

\begin{document}

\title{GEM Übung: \textbf{Blatt 1} Mitschrift}
\date{29. Oktober 2013, 5. November 2013}
\author{Kevin Meyer}
\maketitle

\section*{Zusammenfassung}
\subsection*{Zusammenhänge}

\begin{center}
\begin{tabular}{c|c}
\hline 
magnetisch & elektrisch \\ 
\hline \\
$\theta = w \cdot I \unitof{\si{\ampere}}$ & $U$ \\ \\
$\Phi \unitof{\si{\weber}}=\unitof{\si{\volt\second}} $& $I$\\ \\
$B = \frac{\Phi}{A} (\Phi \bot A)$ & $s = \frac{I}{A}$\\ \\
$\Phi = \iint\limits_A \vec{B} d \vec{A}$ &  \\ \\
$V_m = R_m \cdot\Phi = \int\vec{H} \cdot d \vec{l}$ & $U = R_{el} \cdot I= \int\vec{E} \cdot d \vec{l}$\\ \\
$R_m = \frac{l}{\mu\cdot A}$ & \\ \\
$\mu = \mu_0 \cdot \mu_r$ & \\ \\
\hline
\end{tabular} 
\end{center}

\subsection*{Nützliche Gleichungen}
\begin{align*}
\text{rot} \vec{H} = \vec{s}\\
\oint\vec{H} d \vec{l} = \theta\\
B = \mu\cdot H\\
\text{div}\vec{B}=0\\
\sum\Phi = 0\\
U_i = \frac{d\Psi}{dt}\\
\Psi = \Phi\cdot w
\end{align*}

\newpage

\section{Aufgabe}
\subsection{}
\begin{align*}
\oint Hdl = \Theta\\
2 \cdot H_\delta \cdot\delta = I \cdot w_{sp} \cdot 2\\
H_\delta = \frac{2 \cdot w_{sp} \cdot I}{2 \cdot \delta} = \ldots = \SI{558}{\kilo\ampere\per\meter}\\
B_\delta = \mu_0 H_\delta = \ldots = \SI{0,701}{\tesla}
\end{align*}

\subsection{}
%Zeichnung einfügen! (Siehe Kurzlösung)

\begin{center}
\begin{circuitikz} 
\draw	(0,2) to [R=$R_S$](2,2)
			to [V<=$\Theta_1$](4,2)
			to [R=$R_\delta$](6,2)
			to [R=$R_J$](6,0)
			to [R=$R_\delta$](4,0)
			to [V<=$\Theta_2$](2,0)
			to [R=$R_S$](0,0)
			to [R=$R_J$] (0,2);
\end{circuitikz}
\end{center}

\subsection{}
\begin{align*}
B_\delta = ?\\
B_\delta = \frac{\Phi}{A_\delta}\\
\Phi = \frac{\sum\Theta}{R_{ges}}\\
R_{m Fe} = \frac{l}{\mu_0\mu_r \cdot A}\\
A = A_{ges} \cdot k_{Fe}\\
\vdots
\end{align*}

\subsection{}
\begin{align*}
F_L = l \cdot B_\delta \cdot I_L = \ldots = \SI{2,62}{\newton} \\
M = F_L \cdot l_H = \ldots = \SI{0,262}{\newton\metre}
\end{align*}

\section{Aufgabe}
\subsection{}
\begin{align*}
U(t) = \hat{U} \cdot \cos (\omega t) \\
\hat{U} = 230 V \cdot \sqrt{2}\\
U_i = \frac{d \Psi}{dt} = 2 w_{sp} \frac{d \Phi}{dt} \\
\Phi = \int \frac{U_i}{2 w_{sp}} dt = \int \frac{\hat{U} \cos (\omega t)}{2w_{sp}} dt = \frac{\hat{U}}{2w_{sp} \omega} \sin (\omega t) + C (C=0) \\
\Phi (t) = \frac{\ldots}{\ldots} \cdot \sin (\omega t) \\
\Theta(t) = R_{mag} \cdot\Phi(t)\\
i(t) = \frac{\Theta(t)}{2w_{sp}}\\
\end{align*}

Ende Übung vom 29.10.2013
%\\Anfang Übung vom 05.11.2013

\begin{align*}
B_\delta(t) = \frac{\Phi(t)}{A_\delta} = \ldots = \SI{0,994}{\tesla} \cdot \sin(\omega t)
\end{align*}

\subsection{}
\begin{align*}
F_L = l \cdot B_\delta(t) \cdot I_L = \ldots = \SI{3,98}{\newton} \cdot \sin(\omega t) \\
m_L = F_L \cdot l_H = \ldots = \SI{0.398}{\newton\meter} \cdot \sin(\omega t)\\
\end{align*}

\subsection{}
\begin{align*}
L= \frac{\Psi}{i} = \frac{2\cdot w_{sp}\cdot \Phi}{i} = \SI{0.227}{\henry}
\end{align*}
\end{document}