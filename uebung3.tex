%
% 
% Geschrieben im WS 2013/2014 an der TU München
% von Markus Hofbauer und Kevin Meyer für LaTeX4EI (latex4ei.de)
% Kontakt: latex@kevin-meyer.de oder via Kontaktformular auf http://latex4ei.de

% Dokumenteinstellungen
% ======================================================================	

\documentclass[10pt,a4paper]{article}
\usepackage[utf8]{inputenc}
\usepackage[german]{babel}
\usepackage{scientific}
\usepackage[european]{circuitikz}

\sisetup{per-mode=fraction,output-decimal-marker={,}}

% Nicht neuen, sondern alten Vector benutzen
\let\newvec = \vec
\let\vec = \oldvec

% Dokumentbeginn
% ======================================================================

\begin{document}

\title{GEM Übung: \textbf{Blatt 3} Mitschrift}
\date{12. und 19. November 2013}
\author{Kevin Meyer}
\maketitle

%\section*{Zusammenfassung}

% Aufgabe 1
\section{Aufgabe}
\begin{align*}
2p = 4
\end{align*}

% Aufgabe 2
\section{Aufgabe}
Zeichnung: Siehe Lösung.

% Aufgabe 3
\section{Aufgabe}
\begin{align*}
B_\delta = \frac{\Phi_\text{Pol}}{A_\text{Pol}}\\
\Phi_\text{Pol}= B_\delta \cdot A_\text{Pol} = B_\delta \cdot b_p \cdot l_i\\
b_p = \beta_p \cdot\frac{D_A}{2} \cdot \frac{2\pi}{\SI{360}{\degree}} = \ldots = \SI{0.346}{\meter}\\
\Rightarrow\Phi_\text{Pol} = \SI{138.4}{\milli\weber}
\end{align*}

% Aufgabe 4
\section{Aufgabe}
\begin{align*}
\delta' = k_c \cdot \delta_\text{ges}\\
k_c = k_{c1}\cdot k_{c2} \\
\text{Hier: }\k_{c1} = 1 \text{ Da Stator ohne Nutteilung.}\\
k_c = k_{c2} = \frac{\tau_N}{\tau_n - \gamma\cdot\delta_g}\\
\gamma = \frac{(\frac{b_d}{\delta_g})^2}{5+\frac{b_s}{\delta_g}} = \SI{1,819}{}\\
k_c = \SI{1,263}{}\\
\delta' = \SI{3,79}{\milli\meter}
\end{align*}

% Aufgabe 5
\section{Aufgabe}
\begin{align*}
\oint H \diff l = \Theta_\text{ges}\\
2 \cdot \delta \cdot H_\delta = \Theta_{1,1} \cdot 2\\
\delta' \cdot H_\delta = \Theta_{1,1}\\
\delta' \cdot \frac{B_\delta}{\mu_0} = \Theta_{1,1} = \ldots = \SI{3016}{\ampere}
\end{align*}

% Aufgabe 6
\section{Aufgabe}
\begin{align*}
2\cdot \delta' \cdot H_\delta + 2 \cdot h_Z \cdot H_Z = 2 \cdot \Theta_{1,2}\\
B_\delta\cdot A_{\tau_N} = B_Z \cdot A_Z\\
B_\delta \cdot \tau_N \cdot l_i = B_Z \cdot b_Z \cdot l_i \cdot k_\text{Fe}\\
B_Z = \frac{\tau_N}{b_Z}\cdot \frac{1}{k_\text{Fe}}\cdot B_\delta = \SI{2.93}{\tesla}\\
\text{Ablesen: }H_Z = \SI{40}{\kilo\ampere\per\meter}\\
V_Z = h_Z \cdot H_Z = \ldots = \SI{1440}{\ampere}\\
\Theta_{1,2} = \Theta_{1,1} + V_Z = \ldots = \SI{4456}{\ampere}
\end{align*}

% Aufgabe 7
\section{Aufgabe}
\begin{align*}
\Theta = - \int a \diff l\\
\diff l = \frac{D}{2} \diff \vartheta\\
V(\vartheta) = \Theta(\vartheta) = - \frac{D}{2} \cdot \int a \diff \vartheta\\
\end{align*}

Ende Übung vom 12.11.2013

% Aufgabe 8
\section{Aufgabe}
Zeichung: Siehe Lösung\\

% Aufgabe 9
\section{Aufgabe}
\begin{align*}
\Theta_\text{Nut} = 4 \cdot \SI{114}{\ampere} = \SI{456}{\ampere}\\
\bar{a}_m = \frac{\sum i}{l}\\
\bar{a}_m = \frac{\Theta_\text{Nut}}{\tau_N}\\
\text{oder: } \bar{a}_m = \frac{\Theta_\text{Polteilung}}{\tau_p} = \frac{72 \cdot \si{0,25} \cdot \SI{456}{\ampere}}{D \cdot \pi \cdot \si{0,25}}\\
\ldots = \SI{17,42}{\kilo\ampere\per\meter}\\
V(\vartheta) = - \frac{D}{2} \cdot \int a \cdot \diff \vartheta \rightarrow \text{ Skizze (noch keine Werte)}\\
\abs{V_\text{ma}} = a_m \cdot \frac{\tau_p}{2} = \ldots = \SI{4102}{\ampere}\\
V_\text{2,Polrand} = V_\text{2,max} \cdot \frac{\SI{33}{\degree}}{\SI{45}{\degree}} = \SI{3008}{\ampere}\\
B_\text{2,Polrand} = \mu_0 \cdot \frac{V_\text{2,Polrand}}{\delta'} = \ldots = \SI{1}{\tesla}
\end{align*}

Ende Übung vom 19.11.2013

\end{document}
