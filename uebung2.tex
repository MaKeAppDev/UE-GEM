%
% 
% Geschrieben im WS 2013/2014 an der TU München
% von Markus Hofbauer und Kevin Meyer für LaTeX4EI (latex4ei.de)
% Kontakt: latex@kevin-meyer.de oder via Kontaktformular auf http://latex4ei.de

% Dokumenteinstellungen
% ======================================================================	

\documentclass[10pt,a4paper]{article}
\usepackage[utf8]{inputenc}
\usepackage[german]{babel}
\usepackage{scientific}
\usepackage[european]{circuitikz}

\sisetup{per-mode=fraction,output-decimal-marker={,}}

% Nicht neuen, sondern alten Vector benutzen
\let\newvec = \vec
\let\vec = \oldvec

% Dokumentbeginn
% ======================================================================

\begin{document}

\title{GEM Übung: \textbf{Blatt 2} Mitschrift}
\date{5. November 2013}
\author{Kevin Meyer}
\maketitle

%\section*{Zusammenfassung}

\section{Aufgabe}

\begin{align*}
B \rightarrow \underset{\text{ablesen}}{H} \rightarrow V_\text{mag} 
\Rightarrow \sum V_\text{mag} = \Theta \\ \\
B_\delta = \frac{\Phi}{A_\delta} = \ldots = \SI{1,643}{\tesla} \\
H_\delta = \frac{B_\delta}{\mu} = \ldots = \SI{1307}{\kilo\ampere\per\meter} \\
V_\delta = H_\delta\cdot 2\cdot \delta = \ldots = \SI{5229}{\ampere}\\
B_S = \frac{\Phi}{A_S\cdot k_\text{Fe}} = \ldots = \SI{1,786}{\tesla}\\
H_S: \text{ablesen} \Rightarrow \SI{11,2}{\kilo\ampere\per\meter}\\
V_S = 2\cdot l_S\cdot H_S = \SI{2016}{\ampere}\\
B_J = \frac{\Phi}{A_J\cdot k_\text{Fe}} = \SI{1,563}{\tesla}\\
H_J: \text{ablesen} \Rightarrow \SI{2,95}{\kilo\ampere\per\meter}\\
V_J = 2\cdot l_j\cdot H_J = \SI{708}{\ampere}\\
\Phi = V_\delta + V_S +V_J = \SI{7953}{\ampere}\\ \\
\framebox{\SI{2,55}{\milli\weber}} \\
V_\delta \approx \SI{5800}{\volt}\\
H_S \approx \SI{35}{\kilo\ampere\per\meter}\\
V_S \approx \SI{6300}{\ampere}\\
H_J \approx \SI{8300}{\ampere\per\meter}\\
V_J \approx \SI{1800}{\ampere}\\
\Theta_\text{ges} \approx \SI{13900}{A}
\end{align*}

Graph: Zeichnung siehe Kurzlösung.

\section{Aufgabe}
\begin{align*}
u_i = \frac{\diff \Psi}{\diff t} = \frac{2\cdot w_\text{sp}\cdot \diff\Phi}{\diff t}\\
\Phi(t) = \int \frac{\hat{U}\cdot\cos(\omega t)\cdot H}{2\cdot w_\text{sp}} \diff t = \frac{\hat{U}}{2\cdot w_\text{sp}\cdot \omega} \cdot \sin(\omega t) (+C)\\
\Phi(t) = \SI{2,3}{\milli\weber} \cdot \sin(\omega t)
\end{align*}

\section{Aufgabe}
\begin{align*}
i(t) = ?\\
\end{align*}
\begin{center}
\begin{tabular}{c||c|c|c}
$\omega t$ & $\Phi(\omega t)$ & $\Theta$ & $i$ \\ 
\hline \hline
$\SI{0}{\degree}$ &  &  &  \\ 
\hline 
$\SI{10}{\degree}$ & $\SI{1,15}{\milli\weber}$ & $\SI{2675}{\ampere}$ & $\SI{3,6}{\ampere}$ \\ 
\hline 
\vdots &  &  &  \\ 
\hline 
$\SI{90}{\degree}$ & & &
\end{tabular} 
\end{center}

\section{Aufgabe}

Siehe Lösung

\end{document}