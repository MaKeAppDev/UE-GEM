%
% 
% Geschrieben im WS 2013/2014 an der TU München
% von Markus Hofbauer und Kevin Meyer für LaTeX4EI (latex4ei.de)
% Kontakt: latex@kevin-meyer.de oder via Kontaktformular auf http://latex4ei.de

% Dokumenteinstellungen
% ======================================================================	

\documentclass[10pt,a4paper]{article}
\usepackage[utf8]{inputenc}
\usepackage[german]{babel}
\usepackage{scientific}
\usepackage[european]{circuitikz}
\usepackage{hyperref}

\sisetup{per-mode=fraction,output-decimal-marker={,}}

% Nicht neuen, sondern alten Vector benutzen
\let\newvec = \newvec
\let\newvec = \oldvec

% Dokumentbeginn
% ======================================================================

\begin{document}

\title{GEM Übung: \textbf{Blatt 9} Mitschrift}
\date{28. Januar 2014}
\author{Kevin Meyer}
\maketitle


% Aufgabe 1
\section{Aufgabe}
\begin{align*}
U_1 = \frac{U_N}{\sqrt{3}} = \ldots = \SI{381}{\volt}\\
n_N = \frac{f}{p} = \ldots = \SI{6.25}{\per\second}\\
M_{iN} = \frac{P_N}{\omega_m} = \ldots = \SI{10.19}{\kilo\newton\meter}
\end{align*}

% Aufgabe 2
\section{Aufgabe}
\begin{align*}
M_i = -m \cdot \frac{p}{\omega_\text{el}} \cdot U_1 \left[\frac{U_{ip}}{X_d} \cdot \sin(\vartheta) + \frac{U_1}{2} \left( \frac{1}{X_q} - \frac{1}{X_d} \right) \cdot \sin(2\vartheta)\right]\\
I_{E} = 0 \Rightarrow U_{ip} = 0\\
M_i = \underbrace{-m \cdot \frac{p}{\omega_\text{el}} \cdot \frac{{U_1}^2}{2} \left( \frac{1}{X_q} - \frac{1}{X_d} \right)}_{M_{R\text{max}}} \cdot \sin(2\vartheta)\\
M_{R\text{max}} = \ldots = \SI{2.34}{\kilo\newton\meter}
\end{align*}

% Aufgabe 3
\section{Aufgabe}
\begin{align*}
\vartheta = \SI{-25}{\degree}\\
\abs{\varphi} = \arccos(\SI{0.8}{}) = \SI{36.9}{\degree}\\
\text{übererregt} \Rightarrow \vec I_1 \text{ eilt $U_1$ voraus}\\
\vec I_1 = \SI{437.4}{\ampere} \cdot e^{j \SI{36.9}{\degree}}\\
\abs{I_d} = \SI{387}{\ampere} \quad (\triangleq \SI{3.87}{\centi\meter} \text{, bei $\SI{1}{\centi\meter} \triangleq \SI{100}{\ampere}$})\\
\abs{I_q} = \SI{210}{\ampere}\\
\vec U_{ip} = \vec U_1 - jX_d\cdot \vec I_d - jX_q\cdot \vec I_q\\
\abs{X_d\cdot \vec I_d} = \ldots = \SI{433.4}{\volt}\\
\abs{X_q\cdot \vec I_q} = \ldots = \SI{159.6}{\volt}\\
U_{ip} = \SI{782.5}{\volt}\\
\vec U_{ip} = \SI{782.5}{\volt} \cdot e^{-j\SI{25}{\degree}}
\end{align*}

% Aufgabe 4
\section{Aufgabe}
\begin{align*}
\vec U_1 -jX_q\cdot \vec I_1 \Rightarrow \text{Lage der Polradspannung}\\
U_{ip} = \SI{695}{\volt} \cdot e^{j\SI{28}{\degree}}
\end{align*}

% Aufgabe 5
\section{Aufgabe}
\begin{align*}
I_d = \SI{320}{\ampere}\\
I_q = \SI{235}{\ampere}
\end{align*}

\end{document}