%
% 
% Geschrieben im WS 2013/2014 an der TU München
% von Markus Hofbauer und Kevin Meyer für LaTeX4EI (latex4ei.de)
% Kontakt: latex@kevin-meyer.de oder via Kontaktformular auf http://latex4ei.de

% Dokumenteinstellungen
% ======================================================================	

\documentclass[10pt,a4paper]{article}
\usepackage[utf8]{inputenc}
\usepackage[german]{babel}
\usepackage{scientific}
\usepackage[european]{circuitikz}
\usepackage{hyperref}

\sisetup{per-mode=fraction,output-decimal-marker={,}}

% Nicht neuen, sondern alten Vector benutzen
\let\newvec = \newvec
\let\newvec = \oldvec

% Dokumentbeginn
% ======================================================================

\begin{document}

\title{GEM Übung: \textbf{Blatt 7} Mitschrift}
\date{14. Januar 2014}
\author{Kevin Meyer}
\maketitle


% Aufgabe 1
\section{Aufgabe}
\subsection{}
\begin{align*}
n_N = \frac{f_1}{p} = \frac{\SI{50}{\hertz}}{2} = \SI{1500}{\per\minute}
\end{align*}

\subsection{}
\begin{align*}
X_d = X_{1 \sigma} + X_{1h} = \ldots = \SI{21.8}{\ohm}
\end{align*}
\subsection{}
\begin{circuitikz}
\draw [american currents]
	(0,2)
	to [short, i=$\underline{I}_1$, o-](1, 2) 
	to [R=$R_1$](3.5,2)
	to [L=$jX_{1\sigma}$] (5.5,2)
	to [L=$jX_{1h}$] (7.5,2)
	to [V=$\underline{U}_\text{iP}$] (7.5,0)
	to [short, -o]		(0,0)
	to [open, v^=$\underline{U}_1$]	(0,2);
\end{circuitikz}
\begin{circuitikz}
\end{circuitikz}

\section{Aufgabe}
\subsection{}
\begin{align*}
U_{ip} = \sqrt{2} \cdot M_{21} \cdot \omega \cdot I_2\\
U_{1N} I_{02}\\
M_{21} = \frac{U_{1N}}{\sqrt{2} \cdot I_{02} \cdot \omega} = \ldots = \SI{127.9}{\milli\henry}
\end{align*}

\subsection{}
\begin{align*}
I_K = \frac{U_{1N}}{\newvec Z} = \frac{U_{1N}}{X_d} = \ldots = \SI{166,85}{\ampere}
\end{align*}

\section{Aufgabe}
\subsection{}
\begin{align*}
\abs{I_{N1}} = \frac{S_N}{3 \cdot U_{1N}} = \ldots = \SI{229.1}{\ampere}\\
\varphi = \arccos(\SI{-0.8}{}) = \pm \SI{143.1}{\degree}\\
\varphi = \SI{-143.1}{\degree} \text{ weil Generator übererregt.}\\
\newvec I_{N1} = \abs{I_{N1}} \cdot e^{j\varphi_I} = \SI{229.1}{\ampere} \cdot e^{j + \SI{143.1}{\degree}}
\end{align*}

\subsection{}
\begin{align*}
P_{N1} = \cos\varphi \cdot S_N = \SI{0.8}{} \cdot \SI{2.5}{\mega\volt\ampere} = \SI{-2}{\mega\watt}\\
P_{mN} = P_{N1}
\end{align*}

\subsection{}
\begin{align*}
M_{iN} = \frac{P_{mN}}{\omega_N} = \frac{\SI{-2}{\mega\watt}}{2\pi \cdot n_\text{syn}} = \ldots = \SI{-12.73}{\newton\meter}
\end{align*}

\subsection{}
\begin{align*}
\vec U_{ip} = U_{1} - jX_d \cdot \vec I_1 = \SI{3637}{\volt} - j\SI{21.8}{\ohm} = \SI{6.64}{\kilo\volt} + j\SI{3.99}{\kilo\volt} = \SI{7.747}{\kilo\volt} \cdot e^{j \SI{31}{\degree}}\\
\vartheta = \SI{31.1}{\degree}
\end{align*}

\subsection{}
\begin{align*}
M_K = 3 \cdot \frac{p}{\omega} \cdot \frac{U_{ip} \cdot U_1}{X_d} = \ldots = \SI{24.68}{\kilo\newton\meter}
\end{align*}

\subsection{Test}
\begin{align*}
M_D = - \sin\theta \cdot M_k \rightarrow \theta = \arcsin\left(-\frac{\SI{12.73}{\kilo\newton\meter}}{\SI{24.86}{\kilo\newton\meter}}\right)\\
\vartheta = \SI{31.06}{\degree}
\end{align*}

\subsection{}
\begin{align*}
I_{2N} = \frac{U_{ipN}}{\sqrt{2} \cdot M_{21} \cdot \omega} = \SI{136.3}{\ampere}
\end{align*}

\subsection{}
\begin{align*}
I_{2N} \Rightarrow \Phi_ E\\
I_1 \Rightarrow \Phi_1\\
\epsilon_\text{el} = \varphi_{I2} - \varphi_{I1} = \SI{360}{\degree} - \SI{59}{\degree} - \SI{143}{\degree} = \SI{158}{\degree}\\
\epsilon_m = \frac{\epsilon_\text{el}}{p} = \SI{79}{\degree}
\end{align*}

\end{document}